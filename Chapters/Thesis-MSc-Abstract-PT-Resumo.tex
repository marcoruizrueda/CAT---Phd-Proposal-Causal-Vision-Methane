% #############################################################################
% RESUMO em Português
% !TEX root = ../main.tex
% #############################################################################
% use \noindent in firts paragraph
% reset acronyms
\acresetall
\noindent A variabilidade do metano atmosférico reflete interações complexas entre emissões de zonas húmidas, fontes agrícolas, fugas de combustíveis fósseis e processos de oxidação, contudo as análises atuais baseadas em satélites dependem principalmente de métodos correlacionais que não conseguem distinguir fatores causais de relações confundidas. Esta investigação desenvolve uma estrutura de descoberta causal especificamente projetada para séries temporais ambientais de satélite, integrando observações de metano com classificações de cobertura do solo e dados de reanálise meteorológica através de métodos adaptativos de descoberta causal que avaliam automaticamente a linearidade, lidam com lags temporais, asseguram a estacionaridade e otimizam estruturas de lag para selecionar técnicas analíticas apropriadas para cada par de variáveis. A estrutura estratifica a análise por tipo de ecossistema para caracterizar mecanismos causais específicos de biomas, compara sistematicamente abordagens baseadas em correlação com métodos de descoberta causal para avaliar as suas forças e limitações relativas, e explora a integração de aprendizagem automática para melhorar tanto a precisão preditiva quanto a compreensão mecanística. As relações causais são quantificadas através de redes dirigidas que permitem a atribuição espacial e temporalmente resolvida da variabilidade do metano através de diferentes paisagens e ciclos sazonais, revelando caminhos mecanísticos em vez de associações estatísticas que podem refletir correlações espúrias. Esta investigação em curso estabelece a primeira estrutura causal-consciente da dinâmica global do metano a partir de observações de satélite, demonstra como a inferência causal temporal pode extrair insights acionáveis de dados de observação da Terra, e fornece software de código aberto que permite análise causal reproduzível para monitorização atmosférica e aplicações de política climática.