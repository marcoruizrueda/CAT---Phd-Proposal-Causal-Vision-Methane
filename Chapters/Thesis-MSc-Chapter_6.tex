% #############################################################################
% This is Chapter 6
% !TEX root = ../main.tex
% #############################################################################
% Change the Name of the Chapter i the following line
\fancychapter{Conclusion}
\label{chap:conclusion}
\cleardoublepage

\section{Conclusions}

In conclusion, this doctoral research proposal outlines a novel integration of causal machine learning techniques with satellite-based methane monitoring. Traditional approaches in the field-often centered on correlation and trend analysis-have proven insufficient to fully unravel the complex drivers of atmospheric methane variability. In contrast, a causal inference framework enables the identification of true cause-effect relationships, distinguishing them from spurious associations and thereby providing deeper insights into methane dynamics.

Through an extensive review of the literature, we have established that while current satellite technologies provide unprecedented coverage and resolution of atmospheric methane (Section~\ref{sec:datasets}), the interpretation of these data can be greatly enhanced through causal analysis. Specifically, causal methods can address attribution questions, for example, which environmental factors are genuinely driving changes in methane concentrations?

Our proposed methodology combines several key components: systematic data preprocessing; causal discovery using methods such as Granger tests, transfer entropy, and PCMCI; and rigorous statistical validation. This structured pipeline is designed to extract meaningful, statistically robust causal relationships from high-dimensional and autocorrelated Earth observation datasets.

The expected outcomes of this research span both scientific and methodological domains. Scientifically, we aim to produce the first detailed causal model of methane concentration dynamics, identifying key drivers such as specific land cover types or co-emitted pollutants and quantifying their influence. This model is expected to address pressing questions: to what extent do wetland emissions versus anthropogenic sources dominate regional methane trends? How do meteorological anomalies causally propagate through to methane variability?

Methodologically, the project serves as a case study in the application of causal learning to Earth system science. It aims to set a precedent for similar causal analyses of other climate-related variables, thereby expanding the methodological toolkit available to the environmental science community. Preliminary results already demonstrate the soundness of our approach, showing that it can correct, enrich, and clarify interpretations in ways that traditional statistical analyses cannot.

Importantly, this research has direct practical implications for climate change mitigation and policy. By providing a causal understanding of methane sources and fluctuations, the findings can inform more effective intervention strategies. For example, if temperature-induced wetland emissions are identified as major causal drivers in specific regions, this could motivate targeted adaptation strategies. Conversely, if agricultural activities are shown to causally influence methane variability, it would support sector-specific mitigation policies.

Moreover, the transparency and interpretability of causal models enhance their credibility and usability for a range of stakeholders-scientists, policymakers, and the general public. We move beyond the statement that ``$X$ is correlated with high methane'' to the more actionable claim: ``If we intervene on $X$, we can expect a quantified impact on methane levels, with a defined confidence level.''

Overall, the conclusions of this proposal are that integrating causal inference with satellite remote sensing is not only feasible but highly advantageous for advancing methane research. It addresses well-documented gaps by shifting the analytical focus toward causation, combining multi-source data, handling statistical challenges, and promoting open science practices. If successful, this work will represent a significant step forward in environmental data analysis, establishing a generalizable template for applying causal AI techniques to better understand and manage complex Earth system processes. It will contribute meaningfully both to academic scholarship and to the practical tools available for confronting climate change.

%%%%%

% #############################################################################
\section{Research Challenges}

While our proposed framework is ambitious, it is important to acknowledge its limitations and the boundaries of what the system can conclude.

First, our causal findings are inherently constrained by the quality and resolution of satellite observations detailed in Section~\ref{sec:datasets}. Satellite methane observations provide column-averaged concentrations rather than direct flux measurements, representing a convolution of surface emissions and atmospheric transport processes. A core limitation is that the framework cannot always discern whether a causal link operates through emission strength or atmospheric dynamics. For example, if a causal effect of El Niño on methane levels is detected, it may be unclear whether this is due to changes in wetland emissions or modifications in atmospheric circulation and OH concentrations. Our analysis may detect the existence of a relationship, but attributing its mechanism often requires further domain-specific investigation. This exemplifies the general limitation of inferring process from pattern in observational studies.

Second, the presence of unobserved confounders and the reliance on modeling assumptions impose further constraints. Our framework assumes that by including a comprehensive set of variables, such as land cover and climate drivers, we sufficiently account for major confounding factors. However, there may exist relevant variables that are unobserved or unmeasured (e.g., soil microbial activity, water table depth), which could introduce bias into our causal inferences. If such latent variables drive both methane concentrations and an observed covariate, our methods might incorrectly ascribe causality. To mitigate this, we interpret results in the context of known physical mechanisms and scientific literature, ensuring that any causal claim is critically evaluated for plausibility.

A further limitation involves computational requirements, as detailed in the risk mitigation strategies (Section~\ref{sec:risks_mitigation}). The large-scale data processing necessitates substantial high-performance computing infrastructure.

For the most demanding scenarios, we currently utilize the Deucalion system provided by EuroHPC JU (EuroHPC Portugal)~\cite{eurohpc-deucalion}, and we are actively pursuing access to additional Tier-1 HPC infrastructure, such as the Karolina system at IT4Innovations in the Czech Republic~\cite{it4i-karolina}.

To reduce complexity and runtime, we occasionally aggregate data regionally or temporally. While this approach improves tractability, it inevitably limits spatial and temporal resolution and may obscure fine-scale causal relationships.

Additionally, methods like PCMCI and Granger causality assume that the relevant temporal lag structure is adequately captured. If causal dependencies involve very long-term lags or structural breaks not covered by our lag window, the system may fail to detect them. Thus, the detectable causal timescale is effectively limited to the considered window (e.g., monthly to annual variation).

The interpretation of causal graphs must also be approached with care. A directed edge in a causal graph does not necessarily translate into a straightforward policy lever. For example, a detected link indicating that increased cropland leads to higher methane concentrations requires further domain-specific understanding: Is the effect due to rice paddies, livestock grazing, or a correlated factor? Our system highlights when and where causal relationships may exist, but domain experts are essential for unpacking the mechanisms and translating insights into actionable decisions. In this sense, the framework is a decision-support tool, not an autonomous decision-making engine.

Lastly, uncertainty is unavoidable. We will quantify uncertainty through $p$-values, false discovery rate controls, confidence intervals, and other statistical tools. Nevertheless, no statistical framework can guarantee absolute certainty. Type I errors (false positives) and Type II errors (false negatives) remain possible. A true causal effect may go undetected if the signal-to-noise ratio is low, while an apparent causal link may arise due to coincidental alignment. The system is designed to minimize such errors through rigorous controls, but it cannot eliminate them entirely.

In summary, the system we propose offers a significant advancement in methane analysis by enabling causal interpretation of satellite and environmental data. However, it comes with limitations related to data interpretation, potential unmeasured variables, computational constraints, and the essential role of human expertise in interpreting and applying the results. Recognizing and transparently communicating these limitations is essential for scientific integrity. By doing so, we ensure that end-users-including scientists and policymakers-understand both the confidence and scope of our findings, and we lay the foundation for future work to improve and expand upon this research.

%%%%%
