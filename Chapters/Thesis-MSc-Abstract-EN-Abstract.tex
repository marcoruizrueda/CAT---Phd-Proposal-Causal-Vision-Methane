% #############################################################################
% Abstract Text
% !TEX root = ../main.tex
% #############################################################################
% reset acronyms
\acresetall
% use \noindent in firts paragraph

In this thesis proposal, we embark on an exploratory study aimed at enhancing methane emission monitoring through the integration of Causal Machine Learning (CML) with Generative AI, particularly focusing on satellite imagery analysis. This approach seeks to push beyond the boundaries of traditional machine learning by incorporating counterfactual analysis and the generation of synthetic satellite images via Generative Adversarial Networks (GANs). The intent is to enrich the dataset with diverse environmental scenarios, particularly those rare or extreme conditions not typically captured in existing datasets. While this study is exploratory and results are not guaranteed, the potential of this methodology lies in its ability to possibly improve the accuracy and generalization of methane emission monitoring. Additionally, it holds promise for deepening our understanding of the cause-effect relationships between environmental factors and methane levels, offering a new lens through which to interpret complex ecological data and results, which is crucial for liability purposes.