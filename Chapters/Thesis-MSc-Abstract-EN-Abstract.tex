% #############################################################################
% Abstract Text
% !TEX root = ../main.tex
% #############################################################################
% reset acronyms
\acresetall
% use \noindent in firts paragraph

Methane (\ch{CH_4}) is a short-lived yet powerful greenhouse gas whose emissions arise from a complex interplay of anthropogenic activities and natural processes. Although satellite missions like TROPOMI have enabled global monitoring of atmospheric \ch{XCH_4} at unprecedented spatiotemporal scales, most analyses remain correlational, lacking the tools to disentangle causality in the methane cycle. This proposal introduces a causal framework tailored to satellite-based environmental time series, integrating methane retrievals with dynamic land cover classifications, atmospheric co-constituents, and reanalysis-based meteorological reanalysis data. The framework applies Granger causality, transfer entropy, and PCMCI to systematically uncover direct and mediated influences on methane dynamics, explicitly accounting for lag structures, autocorrelation, and confounding. The analysis is stratified by land cover types, allowing for ecosystem-specific assessments of causal mechanisms, enabling spatially and temporally resolved attribution of methane variability. Our approach accounts for confounding, autocorrelation, and heterogeneity in observational data, enabling the construction of causal graphs that reveal interpretable mechanistic pathways rather than mere associations. By moving beyond trend analysis, the project aims to support more accurate source attribution and provide actionable insights for methane mitigation policy. The proposed framework addresses critical research gaps in Earth observation and advances reproducible methodologies for causal environmental monitoring.


