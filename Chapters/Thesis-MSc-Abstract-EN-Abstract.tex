% #############################################################################
% Abstract Text
% !TEX root = ../main.tex
% #############################################################################
% reset acronyms
\acresetall
% use \noindent in firts paragraph

Atmospheric methane variability reflects complex interactions between wetland emissions, agricultural sources, fossil fuel leaks, and oxidation processes, yet current satellite-based analyses rely primarily on correlational methods that cannot distinguish causal insights from confounded relationships. This research develops a causal discovery framework specifically designed for satellite environmental time series, integrating methane observations with land cover classifications and meteorological reanalysis data through adaptive causal discovery methods that automatically assess linearity, handle temporal lags, ensure stationarity, and optimize lag structures to select appropriate analytical techniques for each variable pair. The framework stratifies analysis by ecosystem type to characterize biome-specific causal mechanisms, systematically compares correlation-based approaches with causal discovery methods to evaluate their relative strengths and limitations, and explores machine learning integration to enhance both predictive accuracy and mechanistic understanding. Causal relationships are quantified through directed networks that enable spatially and temporally resolved attribution of methane variability across different landscapes and seasonal cycles, revealing mechanistic pathways rather than statistical associations that may reflect spurious correlations. This ongoing research establishes the first causal-aware framework of global methane dynamics from satellite observations, demonstrates how temporal causal inference can extract actionable insights from Earth observation data, and provides open-source software that enables reproducible causal analysis for atmospheric monitoring and climate policy applications.
