% #############################################################################
% Abstract Text
% !TEX root = ../main.tex
% #############################################################################
% reset acronyms
\acresetall
% use \noindent in firts paragraph

The radiative forcing caused by methane contributes significantly to the warming of the atmosphere. To unravel the complexities of the global methane cycle, space-based measurements must be methodically analyzed to identify key patterns and underlying factors. This study proposes the integration of causal machine learning (CML) with satellite imagery to advance our capabilities in monitoring and understanding methane emissions. Unlike existing methodologies that depend heavily on associative models, which often struggle to discern the root causes affecting methane detection accuracy and generalization, our approach seeks to uncover the causal relationships that govern methane dynamics on a global scale. This study posits that incorporating soil characteristics and environmental factors into a causal framework could revolutionize our understanding and monitoring capabilities of methane emissions. By leveraging advanced CML techniques and satellite data, the research seeks to unveil the intricate cause-effect dynamics governing methane emissions, potentially leading to more accurate, generalizable, and interpretable methane monitoring systems. This approach not only promises to refine current emission estimates but also to offer novel insights into the environmental impact assessment and mitigation strategies, thereby contributing significantly to the field of environmental monitoring and climate change mitigation.