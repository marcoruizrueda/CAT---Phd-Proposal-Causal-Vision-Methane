%%%%%%%%%%%%%%%%% LIST OF GLOSSARY TERMS  %%%%%%%%%%%%%
% Methane-specific
\newglossaryentry{methane}{
	name={Methane (CH\textsubscript{4})},
	description={Potent greenhouse gas produced by natural (wetlands, termites, seeps) and anthropogenic (fossil fuel extraction, agriculture, waste) sources},
	symbol={}
}

\newglossaryentry{xch4}{
	name={Column-averaged dry-air mole fraction of methane (XCH\textsubscript{4})},
	description={Column-averaged dry-air mole fraction of methane derived from nadir-looking satellites (e.g., TROPOMI, GOSAT)},
	sort={xch4}
}

% Atmospheric chemistry & radiative forcing
\newglossaryentry{oh}{
	name={Hydroxyl radical (OH)},
	description={Primary atmospheric oxidant controlling the lifetime of methane and many pollutants},
	sort={oh}
}

\newglossaryentry{gwp}{
	name={Global Warming Potential (GWP)},
	description={Integrated radiative forcing of a greenhouse gas relative to CO\textsubscript{2} over a specified time horizon (usually 20--100~years)},
	sort={gwp}
}

\newglossaryentry{rf}{
	name={Radiative Forcing (RF)},
	description={Perturbation of Earth’s radiative balance (W\,m\textsuperscript{-2}) caused by natural or anthropogenic drivers},
	sort={rf}
}

\newglossaryentry{aod}{
	name={Aerosol Optical Depth (AOD)},
	description={Column-integrated extinction of solar radiation by aerosol scattering and absorption},
	sort={aod}
}

\newglossaryentry{column_averaged}{
	name={Column-averaged dry-air mole fraction},
	description={Vertical average of a trace-gas concentration with respect to dry air; standard unit for XCH\textsubscript{4}, XCO\textsubscript{2}, etc.},
	sort={column_averaged}
}

% Satellite missions & instruments
\newglossaryentry{tropomi}{
	name={TROPOspheric Monitoring Instrument (TROPOMI)},
	description={Hyperspectral nadir sensor aboard Sentinel-5P providing global daily maps of atmospheric constituents at 5.5$\times$7~km},
	sort={tropomi}
}

\newglossaryentry{gosat}{
	name={Greenhouse Gases Observing Satellite (GOSAT)},
	description={Japanese mission for high-spectral-resolution SWIR/TIR soundings of CO\textsubscript{2} and CH\textsubscript{4}},
	sort={gosat}
}

\newglossaryentry{modis}{
	name={Moderate Resolution Imaging Spectroradiometer (MODIS)},
	description={NASA multispectral instrument on Terra/Aqua providing global land, ocean and atmosphere products},
	sort={modis}
}

\newglossaryentry{wfmd}{
	name={Weighting-Function Modified DOAS (WFMD)},
	description={Retrieval algorithm for XCH\textsubscript{4}/XCO\textsubscript{2}},
	sort={wfmd}
}

\newglossaryentry{remotec}{
	name={Remote sensing retrieval algorithm (RemoTeC)},
	description={Full-physics optimal-estimation suite for trace-gas retrievals},
	sort={remotec}
}

\newglossaryentry{swir}{
	name={Short-Wave Infrared (SWIR)},
	description={Spectral region ($\approx$1.0--2.5\,µm) exploited for atmospheric CH\textsubscript{4}, CO, CO\textsubscript{2} retrievals},
	sort={swir}
}

\newglossaryentry{sentinel5p}{
	name={Sentinel-5 Precursor (S5P)},
	description={ESA platform carrying TROPOMI (launched 2017) for operational atmospheric composition monitoring},
	sort={sentinel5p}
}

% Causal-inference toolbox
\newglossaryentry{causal_inference}{
	name={Causal inference},
	description={Statistical framework for estimating directional cause--effect relationships in observational data},
	sort={causal_inference}
}

\newglossaryentry{granger_causality}{
	name={Granger causality (GC)},
	description={Test where X “Granger-causes” Y if past X improves the prediction of Y beyond past Y alone},
	sort={granger_causality}
}

\newglossaryentry{transfer_entropy}{
	name={Transfer Entropy (TE)},
	description={Information-theoretic measure of directed coupling capturing non-linear dependencies in time series},
	sort={transfer_entropy}
}

\newglossaryentry{pcmci}{
	name={PC algorithm with Momentary Conditional Independence (PCMCI)},
	description={Scalable causal discovery method for multivariate time series},
	sort={pcmci}
}

\newglossaryentry{confounding}{
	name={Confounding},
	description={Bias arising from variables that affect both the putative cause and effect},
	sort={confounding}
}

\newglossaryentry{autocorrelation}{
	name={Autocorrelation},
	description={Correlation of a variable with its own lagged values},
	sort={autocorrelation}
}

\newglossaryentry{stationarity}{
	name={Stationarity},
	description={Property of a time series whose statistical moments do not change with time},
	sort={stationarity}
}

% Biogeochemical processes
\newglossaryentry{methanogenesis}{
	name={Methanogenesis},
	description={Anaerobic microbial pathway producing CH\textsubscript{4} from H\textsubscript{2}/CO\textsubscript{2}, acetate or methylated compounds},
	sort={methanogenesis}
}

\newglossaryentry{methanotrophic}{
	name={Methanotroph},
	description={Aerobic or anaerobic bacteria that oxidise methane as an energy source},
	sort={methanotrophic}
}

\newglossaryentry{wetlands}{
	name={Wetlands},
	description={Water-saturated ecosystems and the largest natural CH\textsubscript{4} source},
	sort={wetlands}
}

\newglossaryentry{enteric_fermentation}{
	name={Enteric fermentation},
	description={Microbial digestion in ruminants producing CH\textsubscript{4} as a by-product},
	sort={enteric_fermentation}
}

\newglossaryentry{biomass_burning}{
	name={Biomass burning},
	description={Combustion of vegetation in wildfires, agriculture and energy use; emits CH\textsubscript{4}, CO, NO\textsubscript{x}, aerosols},
	sort={biomass_burning}
}

\newglossaryentry{ebullition}{
	name={Ebullition},
	description={Bubble-mediated gas release from sediments or water columns},
	sort={ebullition}
}

% Land-cover classes
\newglossaryentry{croplands}{
	name={Croplands},
	description={Cultivated fields for annual or perennial crops; anthropogenic CH\textsubscript{4} and N\textsubscript{2}O sources},
	sort={croplands}
}

\newglossaryentry{savannas}{
	name={Savannas},
	description={Seasonally dry grasslands with scattered trees; intermediate natural CH\textsubscript{4} emissions},
	sort={savannas}
}

\newglossaryentry{shrublands}{
	name={Shrublands},
	description={Ecosystems dominated by shrubs and low woody vegetation; typically low CH\textsubscript{4} flux},
	sort={shrublands}
}

\newglossaryentry{urban_areas}{
	name={Urban areas},
	description={Built environments with impervious surfaces; indirect CH\textsubscript{4} sources (landfills, leaks)},
	sort={urban_areas}
}

% Statistics, units & technical terms
\newglossaryentry{bottom_up}{
	name={Bottom-up estimates},
	description={Emission estimates derived from activity data, land-surface models and inventories},
	sort={bottom_up}
}

\newglossaryentry{top_down}{
	name={Top-down estimates},
	description={Inverse modelling constrained by atmospheric concentration observations},
	sort={top_down}
}

\newglossaryentry{tg_ch4_yr}{
name={Teragrams of methane per year (Tg\,CH\textsubscript{4}\,yr\textsuperscript{-1})},
description={Standard unit for annual global CH\textsubscript{4} emissions; 1~Tg $=10^{12}$~g},
sort={tg_ch4_yr}
}

\newglossaryentry{ppb}{
	name={Parts per billion (ppb)},
	description={Mole fraction of $10^{-9}$; standard unit for atmospheric CH\textsubscript{4}},
	sort={ppb}
}

\newglossaryentry{netcdf}{
	name={Network Common Data Form (NetCDF)},
	description={Self-describing binary format for array-oriented scientific data},
	sort={netcdf}
}

\newglossaryentry{gee}{
	name={Google Earth Engine (GEE)},
	description={Cloud platform for planetary-scale geospatial analysis},
	sort={gee}
}

% Special categories
\newglossaryentry{super_emitters}{
	name={Super-emitters},
	description={Disproportionately large methane point sources detectable by high-resolution satellites},
	sort={super_emitters}
}

\newglossaryentry{la_nina}{
	name={La Niña},
	description={Cold phase of the El Niño--Southern Oscillation influencing global CH\textsubscript{4} budgets},
	sort={la_nina}
}

\newglossaryentry{eddy_covariance}{
	name={Eddy covariance},
	description={Micrometeorological method measuring surface--atmosphere fluxes at high frequency},
	sort={eddy_covariance}
}

\newglossaryentry{inverse_modeling}{
	name={Inverse modelling},
	description={Technique that infers surface emissions from atmospheric observations and transport models},
	sort={inverse_modeling}
}

\newglossaryentry{quality_flags}{
	name={Quality flags},
	description={Per-pixel metadata in satellite products for screening unreliable retrievals},
	sort={quality_flags}
}

\newglossaryentry{counterfactual}{
	name={Counterfactual},
	description={Concept from the potential outcomes framework defining causation in terms of hypothetical alternative scenarios; $X$ causes $Y$ if, holding all else constant, intervening to change $X$ would change the distribution of $Y$},
	sort={counterfactual}
}

\newglossaryentry{interventional_perspective}{
	name={Interventional perspective},
	description={View of causation emphasizing that causal relationships are those that remain stable under deliberate manipulations or interventions on the cause variable},
	sort={interventional_perspective}
}

\newglossaryentry{temporal_precedence}{
	name={Temporal precedence principle},
	description={Principle stating that causes must precede their effects in time, with the relevant time scale depending on the system under study},
	sort={temporal_precedence}
}

\newglossaryentry{causal_markov_condition}{
	name={Causal Markov condition},
	description={Assumption that, conditional on its direct causes (parents in a causal graph), a variable is independent of its non-descendants},
	sort={causal_markov_condition}
}

\newglossaryentry{faithfulness_assumption}{
	name={Faithfulness assumption},
	description={Assumption that all conditional independencies in the observed data correspond to those implied by the true causal graph, excluding accidental independencies due to parameter cancellations},
	sort={faithfulness_assumption}
}

\newglossaryentry{predictive_causality}{
	name={Predictive causality},
	description={Operational definition of causality in time series where $X$ is said to cause $Y$ if past values of $X$ improve the prediction of $Y$ beyond using past $Y$ alone},
	sort={predictive_causality}
}

%%%%%%%%%%%%%%%%% LIST OF SYMBOLS  %%%%%%%%%%%%%
\newglossaryentry{diam0}{%
	name={Initial diameter (\ensuremath{D_0})},
	description={Initial particle diameter},
	symbol={\ensuremath{\mu\text{m}}},
	type=symbols
}

\newglossaryentry{surfarea}{%
	name={Surface area (\ensuremath{A_s})},
	description={Particle surface area},
	symbol={\ensuremath{\mu\text{m}^2}},
	type=symbols
}
