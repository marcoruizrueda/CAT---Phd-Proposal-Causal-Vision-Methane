% #############################################################################
% This is Chapter 3
% !TEX root = ../main.tex
% #############################################################################
% Change the Name of the Chapter i the following line
\fancychapter{Methodology}
\cleardoublepage
% The following line allows to ref this chapter
\label{chap:methodology}


\section{Methodology}

This section describes the methodological strategy developed and implemented over the first two years of the PhD project and the metholodogy of the current causal-guided study. It integrates theoretical foundations in causal inference, remote sensing-based environmental monitoring, and computational modeling with a structured, staged process involving framework validation, dataset selection, and large-scale satellite time series analysis. The methodology is designed to meet the scientific goals of uncovering robust, interpretable, and spatiotemporally resolved causal relationships governing methane variability across diverse land cover and climatic contexts. It includes both exploratory and confirmatory phases, aiming for scalability, reproducibility, and alignment with Earth observation standards.

\subsection{Research Design}

\textbf{Overall Approach:}
This research adopts a rigorous quantitative approach to analyze the cause-effect dynamics of atmospheric methane using satellite-based time series data. Causal analysis, by its nature, demands precise modeling of temporal dependencies, directional interactions, and confounding control. These requirements are best satisfied through quantitative, data-driven methods. The use of statistical causal discovery techniques allows this study to go beyond mere correlations, focusing instead on uncovering mechanisms that plausibly generate observed methane variability. Qualitative or mixed-methods approaches, while valuable in contextual or policy analysis, are insufficient for extracting statistically supported, temporally resolved causal pathways from the scale and granularity of satellite datasets. The chosen quantitative approach aligns with the research aim of uncovering statistically valid, generalizable causal structures within large spatiotemporal datasets, which would be impractical to investigate using qualitative techniques or surveys. The decision to employ a quantitative framework is rooted in the need for reproducibility, scalability, and precision in assessing multivariate relationships across space and time. Quantitative techniques are particularly suited for integrating large-scale geospatial data with statistical causal inference models, thereby enabling the isolation of robust and generalizable patterns from noisy environmental observations.

\textbf{Specific Design:}
The study follows a longitudinal observational design using high-resolution remote sensing data spanning December 2017 to 2024. This initial period corresponds to the baseline analysis phase, during which the framework is applied to reproduce and extend the reference study results before introducing novel contributions. This approach permits temporal assessments of methane variability in response to atmospheric constituents and land cover dynamics. The design is non-experimental yet analytic, enabling retrospective investigation of spatiotemporal causality in methane concentration shifts without manipulating environmental conditions.

\textbf{Theoretical Framework:}
The research is theoretically anchored in the field of causal discovery and environmental systems modeling. It integrates concepts from dynamical systems, information theory, and statistical causality. Specifically, the adopted causal inference framework includes Granger causality for linear temporal dependencies, Transfer Entropy for non-linear directional information flow, and PCMCI for conditioning on multivariate lags and controlling for autocorrelation and confounders. This theoretical integration ensures that causal pathways are explored with methodological robustness and interpretability.

\subsection{Preliminary Work and Framework Validation}

Prior to full deployment of the causality framework on large-scale satellite datasets, the first phase of the research involved:

\begin{itemize}
  \item A thorough \textbf{literature review} to assess the state-of-the-art in satellite-based methane monitoring and causal inference in environmental sciences. This review also incorporated an investigation into Machine Learning applications for methane detection, considering existing ML-ready datasets and exploring both classic and deep learning methodologies.
    \item An investigation of existing satellite methane datasets, encompassing a comprehensive review of all satellites equipped for methane monitoring (e.g., Sentinel-5P, GOSAT, GHGSat, MethaneSAT, MERLIN, etc.), their respective sensors and payloads (e.g., TROPOMI, TANSO-FTS, imaging spectrometers, LIDAR), and the key differences in their measurement capabilities, spatial and temporal resolutions, and data access policies.
  \item Development and implementation of a \textbf{modular causal inference framework}, coded in Python, allowing integration with satellite time series workflows.
  \item \textbf{Validation with synthetic and tiny observational datasets}, including artificial time series with known causal structures and curated regional methane series. This stage was essential to evaluate the sensitivity, false discovery rates, and temporal resolution limitations of the applied methods.
  \item Establishment of data ingestion and preprocessing pipelines, particularly Google Earth Engine scripts for dynamic querying of MODIS and Sentinel-5P collections.
\end{itemize}

These early activities were instrumental in refining model parameters, understanding the behavior of each method under constrained conditions, and optimizing computational strategies for subsequent large-scale applications. Specifically, the synthetic data experiments helped quantify false positive and false negative rates under controlled noise and sampling scenarios, which led to the adoption of stricter thresholding and bootstrapping routines in the linearity assessment and Transfer Entropy implementation. Validation with regional methane datasets also revealed the necessity of preprocessing modules for temporal alignment and variable normalization, which were subsequently integrated into the preprocessing pipeline. These insights directly informed adjustments to the data cleaning routines, causal lag selection strategies, and the parametrization of significance thresholds in the PCMCI framework.

\subsection{Data Sources}

\textbf{Sampling Strategy:}
Purposive sampling was adopted to ensure inclusion of data sources with high temporal resolution, global spatial coverage, and proven reliability in atmospheric remote sensing. This strategy also accounted for geographic and ecological diversity by using the whole earth (except for water), ensuring representation across distinct climatic zones and land cover classes such as wetlands, croplands, forests, and shrublands. This stratification is crucial for enabling land cover-specific causal inferences.

\textbf{Sample Size:}
The current dataset includes millions of spatiotemporal observations across variables such as methane (XCH$_4$), CO, NO$_2$, and land cover classes. This data scale enables rigorous multivariate and lag-aware analysis with high statistical power.

\textbf{Inclusion and Exclusion Criteria:}
Inclusion was based on spatial location within land-covered areas (e.g., wetlands, croplands), signal \gls{quality_flags}, and complete co-observation of methane and at least two covariates. Exclusion applied to water bodies, flagged pixels with high cloud probability, or measurements with reported uncertainty above acceptable thresholds acording to the Algorithm Theoretical Basis Document (ATBD) of the TROPOMI/WFMD data product. The data sources are described in the Section \ref{sec:datasets}.

\subsection{Data Collection}

\textbf{Methods:}
Data acquisition for the MODIS dataset is performed through automated GEE pipelines via the GEE Python API and is locally refined through custom processing scripts. The acquisition of the TROPOMI/WFMD \gls{netcdf} files was manually downlaoded from a dedicated portal. Data processing scripts incorporated batch streaming, resume-safe modes, and chunked retrieval to manage memory usage and avoid API quotas.

\textbf{Procedures:}
\begin{enumerate}
  \item Extraction of methane time series and ancillary variables for each grid cell in the study regions.
  \item Integration of MODIS-derived categories matched with geographical regions and timestamps.
  \item Assignment of land cover labels to each observation via a median filter on surrounding MODIS/ESA land cover classifications.
\end{enumerate}

\textbf{Instruments:}
Remote sensing instruments included TROPOMI for trace gases and MODIS for surface reflectance and vegetation metrics. These instruments have documented accuracy through inter-comparison studies and validation campaigns.

\textbf{Ethical Considerations:}
All data utilized are public domain and do not involve human subjects. Ethical rigor was upheld through transparent documentation of the full data processing workflow, from acquisition through analysis. Version-controlled repositories were maintained to track changes in code and data, allowing full reproducibility. Data traceability was further ensured by logging metadata for each satellite product, including access timestamps, processing steps, and geographic regions involved. The project adheres to the FAIR data principles, promoting the findability, accessibility, interoperability, and reusability of all materials produced.


\subsection{Justification and Limitations}

Having outlined our methodological framework, we now proceed to critically evaluate the proposed approach. This section highlights its key strengths, acknowledges inherent limitations, and justifies the selection of methods over potential alternatives. For a comprehensive overview of the anticipated challenges and their mitigation strategies, refer to Section~\ref{sec:risks_mitigation}, "Risks and Mitigation Strategies."

\textbf{Strengths:}
The proposed methodology is characterized by its robustness, interpretability, and computational scalability. A key strength lies in the triangulation of causal relationships through the application of multiple complementary causal inference methods (Granger Causality, Transfer Entropy, and PCMCI), which enhances reliability and confidence in the derived findings. The stratification of the analysis by land cover types significantly enhances ecological validity, allowing for granular, policy-relevant insights into methane dynamics in diverse environmental contexts.

\textbf{Limitations:}
Despite its strengths, the methodology acknowledges certain limitations. A primary challenge involves the resolution mismatch between heterogeneous datasets, such as the higher spatial resolution of TROPOMI methane retrievals compared to MODIS land cover products. This discrepancy could potentially introduce spatial smoothing or misalignment during variable integration. To mitigate these effects, rigorous preprocessing will be implemented, including resampling data to a common spatial grid to ensure accurate co-location. Additionally, advanced spatial and temporal filtering strategies, such as median filtering and sophisticated temporal interpolation, will be applied to effectively address data gaps and reduce noise stemming from differing acquisition frequencies. These measures are crucial for preserving the integrity of underlying spatiotemporal trends and preventing the artificial inflation of statistical relationships. Another limitation is the inherent complexity of disentangling multiple causal drivers in a highly interconnected environmental system, particularly when potential unobserved confounders may exist.

\textbf{Alternative Methods:}
While approaches such as Bayesian networks or structural equation modeling were considered, they were not adopted due to challenges with high-dimensional temporal dependencies and data volume. The selected methods align more closely with the objectives and nature of satellite environmental time series.

%%%%%%%%%%%%%%%%%%%
\subsection{Methodology of the Causal Framework}
\label{sec:metho_causal_framework}

This methodology outlines a causal framework for transforming raw multi-sensor satellite observations into robust causal statements about methane (\ch{XCH4}) variability, addressing the spatiotemporal and statistical challenges of satellite-based causal discovery. The framework integrates two interconnected pipelines: a data processing pipeline to prepare high-quality, stratified datasets and a causal analysis pipeline to infer and interpret causal relationships. These pipelines support research Objectives 1-7 (Section \ref{sec:research-objectives}) by ensuring data quality, identifying causal drivers, and providing interpretable results.

\begin{figure}[htbp]
\centering
\begin{tikzpicture}[
    node distance=2cm,
    box/.style={rectangle, draw, rounded corners, minimum width=3cm, minimum height=1.2cm, align=center, fill=blue!10},
    data/.style={rectangle, draw, rounded corners, minimum width=2.5cm, minimum height=0.8cm, align=center, fill=green!10},
    arrow/.style={->, thick, blue},
    stepnode/.style={rectangle, draw, rounded corners, minimum width=2cm, minimum height=0.6cm, align=center, fill=orange!10}
]
% Input data
\node[data] (input) at (0,0) {WFMD\\NetCDF Files};
% Step 1
\node[box] (step1) at (4,0) {Step 1: Data Extraction\\Parallel NetCDF Processing};
\node[data] (data1) at (8,0) {methane\_all01.txt};
% Step 2
\node[box] (step2) at (4,-2.5) {Step 2: Land Cover\\Classification};
\node[data] (data2) at (8,-2.5) {methane\_all02.txt};
% Step 3
\node[box] (step3) at (4,-5) {Step 3: Histogram\\Generation \& Means};
\node[data] (data3) at (8,-5) {Continent-Season\\Histograms \& matrix.csv};
% Step 4
\node[box] (step4) at (4,-7.5) {Step 4: Heatmap\\Visualization};
\node[data] (data4) at (8,-7.5) {Matrix Heatmaps};
% Step 5
\node[box] (step5) at (4,-10) {Step 5: Land Cover\\Deviation Analysis};
\node[data] (data5) at (8,-10) {Deviation Heatmaps};
% Algorithm labels
\node[stepnode] (alg1) at (0,-1.2) {Algorithm 1 \ref{alg:step01_extract}};
\node[stepnode] (alg2) at (0,-3.7) {Algorithm 2 \ref{alg:step02_add_lc}};
\node[stepnode] (alg3) at (0,-6.2) {Algorithm 3 \ref{alg:step03_histos}};
\node[stepnode] (alg4) at (0,-8.7) {Algorithm 4 \ref{alg:step04_heatmap}};
\node[stepnode] (alg5) at (0,-11.2) {Algorithm 5 \ref{alg:step05_health_check}};
% Arrows
\draw[arrow] (input) -- (step1);
\draw[arrow] (step1) -- (data1);
\draw[arrow] (data1) -- (step2);
\draw[arrow] (step2) -- (data2);
\draw[arrow] (data2) -- (step3);
\draw[arrow] (step3) -- (data3);
\draw[arrow] (data3) -- (step4);
\draw[arrow] (step4) -- (data4);
\draw[arrow] (data4) -- (step5);
\draw[arrow] (step5) -- (data5);
% Algorithm references
\draw[arrow, dashed] (alg1) -- (step1);
\draw[arrow, dashed] (alg2) -- (step2);
\draw[arrow, dashed] (alg3) -- (step3);
\draw[arrow, dashed] (alg4) -- (step4);
\draw[arrow, dashed] (alg5) -- (step5);
% Legend
\node[draw, rounded corners, fill=white, thick, inner sep=5pt] (legend) at (12,-3) {
    \begin{tikzpicture}[
        box/.style={rectangle, draw, rounded corners, minimum width=2.5cm, minimum height=0.8cm, align=center, fill=blue!10},
        data/.style={rectangle, draw, rounded corners, minimum width=2.5cm, minimum height=0.8cm, align=center, fill=green!10},
        stepnode/.style={rectangle, draw, rounded corners, minimum width=2.5cm, minimum height=0.8cm, align=center, fill=orange!10}
    ]
        \node[box]      (l1) at (0,0)   {Processing Steps};
        \node[data]     (l2) at (0,-1)  {Data Files};
        \node[stepnode] (l3) at (0,-2)  {Algorithm Labels};
    \end{tikzpicture}
};
\end{tikzpicture}
\caption{Data processing pipeline for methane analysis, transforming raw WFMD NetCDF files into stratified datasets through extraction, classification, histogram generation with means computation, visualization, and land cover deviation analysis.}
\label{fig:pipeline_workflow}
\end{figure}

The data processing pipeline, depicted in Figure~\ref{fig:pipeline_workflow}, transforms raw NetCDF files from the Sentinel-5P TROPOMI/WFMD methane product~\cite{TROPOMI_WFMD_PUG} into structured datasets stratified by region and land cover. The extraction process begins with temporal standardization, converting satellite timestamps to Modified Julian Days (MJD) through either Unix epoch conversion  
\[
\mathrm{MJD} = \frac{t}{86400} + 40587.0
\]  
(for numerical timestamps) or datetime64 conversion  
\[
\mathrm{MJD} = \frac{t - \mathrm{MJD}_{\mathrm{epoch}}}{\Delta t_{\mathrm{day}}}
\]  
(for datetime objects), where $\mathrm{MJD}_{\mathrm{epoch}} = \text{1858-11-17}$. This creates a continuous time index essential for temporal alignment.  

Key atmospheric variables are extracted into a matrix $\mathbf{X} \in \mathbb{R}^{n \times 8}$ containing index, MJD, coordinates, \ch{XCH4}, associated uncertainties, and \ch{XCO}. Parallel processing enables scalable integration through chunked array decomposition  
\[
\mathbf{X} = \bigcup_{k=1}^{K} \mathbf{X}_k, \quad K = \left\lceil \frac{N}{\mathrm{chunk\_size}} \right\rceil ,
\]
thereby supporting Objective~1.  

Land cover classification integrates MODIS LC\_Type1 data through spatial–temporal queries  
\[
\mathrm{LC}(x,y,t) = \mathrm{median}(\mathrm{img.sampleRegions}(x,y)),
\]  
where $\mathrm{img} \in \mathcal{C}$ (MODIS/061/MCD12Q1 filtered to $t \pm 183$~days). Non-terrestrial pixels are filtered using  
\[
\mathbf{X}_{\mathrm{land}} = \{ \mathbf{x}_i \in \mathbf{X} \mid \mathrm{LC}_i \notin \{0,17\} \},
\]  
creating land-cover-specific subsets $\mathbf{X}_{\mathrm{cropland}}, \mathbf{X}_{\mathrm{wetland}}, \ldots$ to enable stratified causal analysis (Objective~4).  

Spatiotemporal aggregation follows seasonal masks defined by MJD boundaries $\mathcal{S} = \{s_1, \dots, s_{12}\}$, where  
\[
\mathrm{Season}_k = [s_k, s_{k+1}] \cup [s_{k+4}, s_{k+5}], \quad k \in \{0,1,2,3\},
\]  
and all timepoints for $k=4$. Continent–season aggregates are computed as  
\[
\overline{\mathrm{XCH}_4}^{(c,k)} = \frac{1}{|\mathcal{I}_{c,k}|} \sum_{i \in \mathcal{I}_{c,k}} \mathrm{XCH}_4^{(i)},
\]  
with $\mathcal{I}_{c,k} = \{ i \mid \mathbf{x}_i \in \mathrm{Continent}_c \cap \mathrm{Season}_k \}$. Histogram binning $P(\mathrm{XCH}_4 \mid c,k)$ uses adaptive bandwidth  
\[
h = 0.9 \cdot \min(\sigma, \mathrm{IQR}/1.34) \cdot n^{-1/5}
\]  
to quantify land-surface influences (Objective~2).  

Anomaly detection employs matrix decomposition  
\[
\mathbf{M} = \mathbf{U} \mathbf{\Sigma} \mathbf{V}^\top
\]  
where $\mathbf{M} = [\overline{\mathrm{XCH}_4}^{(c,k)}]$. Anomalies are isolated via  
\[
\mathbf{M}_{\mathrm{anom}} = \mathbf{u}_1 \sigma_1 \mathbf{v}_1^\top
\]  
to identify regions exceeding $\pm 2\sigma$ residuals. Land-cover deviation analysis computes class-specific anomalies  
\[
\delta\mathrm{LC}^{(c,k,l)} = \overline{\mathrm{XCH}_4}^{(c,k,l)} - \overline{\mathrm{XCH}_4}^{(c,k,\mathrm{all})}
\]  
for $l \in \{\mathrm{classes\ 1\!-\!16}\}$, generating heatmaps $\mathbf{H} = [\delta\mathrm{LC}^{(c,k,l)}]$ that distinguish retrieval artifacts from true emissions (Objective~5). Each stage produces intermediate outputs ensuring modularity and reproducibility.  

The causal analysis pipeline incorporates additional atmospheric variables (\ch{CO}, \ch{NO2}, \ch{H2O}, AOD) with temporal harmonization  
\[
\mathbf{Z}_t^{\mathrm{align}} = \mathbf{W} \mathbf{Z}_t, \quad \mathbf{W}_{ij} = \exp\!\left( -\frac{(t_i - t_j)^2}{2\tau^2} \right), \quad \tau = 7\ \mathrm{days}.
\]  
Quality control removes high-uncertainty observations, while seasonal–trend decomposition and differencing  
\[
\nabla_L \mathrm{XCH}_4^{(t)} = \mathrm{XCH}_4^{(t)} - \mathrm{XCH}_4^{(t-L)}, \quad L=365
\]  
achieve stationarity, verified via Augmented Dickey–Fuller tests ($p<0.01$). Relationships are assessed through Mutual Information  
\[
I(X;Y) = \sum p(x,y) \log \frac{p(x,y)}{p(x)p(y)}
\]  
and correlation metrics, guiding method selection.  


Causal inference in this framework relies on three complementary approaches. First, Granger causality, given by 

\[
F_{\mathrm{Granger}} = \frac{(\mathrm{RSS}_{\mathrm{null}} - \mathrm{RSS}_{\mathrm{full}})/p}{\mathrm{RSS}_{\mathrm{full}}/(T - 2p)},
\]  
tests for linear directional influence between time series. Second, transfer entropy,  
\[
\mathcal{T}_{Y \to X} = \sum p(x_{t+1},x_t,y_t) \log \frac{p(x_{t+1} \mid x_t, y_t)}{p(x_{t+1} \mid x_t)},
\]  
quantifies nonlinear information flow by measuring how much knowing the past of $Y$ improves predictions of $X$. Third, the PCMCI algorithm detects conditional independence via  
\[
\phi_{\mathrm{CI}} = \max_k |\mathrm{ParCorr}(X_t, Y_{t-\tau} \mid \mathbf{Z}^k)|,
\]  
while explicitly addressing autocorrelation effects. Multiple testing corrections are applied across all methods to ensure robustness in high-dimensional settings.

Results are expressed as Directed Acyclic Graphs (DAGs) with Shapley-based attribution  
\[
\phi_j = \sum_{\mathcal{S} \subseteq \mathcal{J} \setminus \{j\}}
\frac{|\mathcal{S}|! \,(|\mathcal{J}| - |\mathcal{S}| - 1)!}{|\mathcal{J}|!}
\left[ f(\mathbf{x}_{\mathcal{S} \cup \{j\}}) - f(\mathbf{x}_{\mathcal{S}}) \right]
\]  
quantifying driver importance (Objective~7). Local surrogate models  
\[
\mathcal{L}(f,g,\pi_x) = \sum (f(z) - g(z))^2 + \Omega(g)
\]  
provide instance-level explanations, while spatial–temporal visualizations contextualize findings. This framework delivers land-cover-specific time series $\{\mathbf{X}_{\mathrm{LC}_i}\}_{i=1}^{17}$ (Objective~4), heatmaps $\mathbf{H}$ identifying emission sources (Objective~5), and DAGs with $\phi_j$-weighted edges for policy attribution (Objective~6), establishing reproducible causal attribution that addresses satellite data challenges.  

This ongoing research effort evolves with new datasets and methodological advances, promoting causal analysis in operational Earth observation (Objective~7). The modular design supports integration of additional drivers such as wetland maps, refining scientific standards for methane attribution aligned with ESA and IPCC priorities.


%%%%%%%%%%%%%%%%%%%

\section{Datasets}
\label{sec:datasets}

\subsection{TROPOMI/WFMD \texorpdfstring{XCH\textsubscript{4}}{XCH4} Dataset}

The primary dataset utilized in this study originates from the TROPOspheric Monitoring Instrument (TROPOMI) aboard the Sentinel-5 Precursor satellite. Methane retrievals are performed using the Weighting Function Modified Differential Optical Absorption Spectroscopy (WFMD) algorithm developed by the University of Bremen \cite{Schneising2019, Schneising2023}. This product offers high-resolution, globally distributed atmospheric methane observations, providing critical data to analyze spatial and temporal trends in methane concentrations.

\subsubsection{Satellite and Instrument Description}
TROPOMI is an imaging spectrometer onboard the Sentinel-5 Precursor satellite, operated by the European Space Agency (ESA). The satellite was launched in October 2017 and maintains a Sun-synchronous orbit, ensuring daily global coverage. TROPOMI measures solar radiation that is backscattered by the Earth's atmosphere and surface, covering spectral bands from ultraviolet (UV) to short-wave infrared (SWIR) \cite{Veefkind2012}.

\subsubsection{WFMD Retrieval Algorithm}
The WFMD algorithm retrieves column-averaged dry-air mole fractions of methane, denoted as XCH\textsubscript{4}, by analyzing spectral radiances in the SWIR bands. Specifically, the algorithm exploits the absorption characteristics of methane near 2.3~\textmu m. XCH\textsubscript{4} is estimated through the following relationship:

\begin{equation}
\mathrm{XCH}_4 = \frac{\int_{p_\mathrm{surf}}^{p_\mathrm{top}} q_{\mathrm{CH}_4}(p) \cdot \mathrm{AMF}(p) \, dp}{\int_{p_\mathrm{surf}}^{p_\mathrm{top}} q_{\mathrm{dry}}(p) \cdot \mathrm{AMF}(p) \, dp}
\end{equation}

where $q_{\mathrm{CH}_4}(p)$ and $q_{\mathrm{dry}}(p)$ represent the vertical profiles of the mixing ratios of methane and dry air, respectively, and $\mathrm{AMF}(p)$ denotes the air mass factor, which accounts for the effects of atmospheric scattering and absorption \cite{Schneising2019, Schneising2023}.

\subsubsection{Dataset Specifications}
The WFMD TROPOMI XCH\textsubscript{4} dataset is characterized by the following properties:

\begin{itemize}
    \item \textbf{Temporal Coverage}: From October 2017 to December 2024 (still processing 2025) with daily global observations.
    \item \textbf{Spatial Resolution}: Approximately 7~$\times$~7~km\textsuperscript{2} at nadir, improved to about 5.5~$\times$~7~km\textsuperscript{2} after August 2019 \cite{Schneising2023}.
    \item \textbf{Variables Included}: Column-averaged methane concentration (XCH\textsubscript{4}), retrieval uncertainties, carbon monoxide (CO), geolocation coordinates, and auxiliary metadata (e.g., surface albedo, cloud fraction).
    \item \textbf{Quality Assurance}: Stringent quality control is applied through multiple flags related to cloud coverage, aerosol loading, viewing geometry, and instrument anomalies. Observations exceeding a cloud fraction threshold (typically 0.2), exhibiting high aerosol optical depths, or associated with anomalous retrieval errors are filtered out to ensure data robustness.
\end{itemize}

\subsubsection{Data Accessibility and Format}
TROPOMI / WFMD XCH\textsubscript{4} data are publicly available in NetCDF format via the Carbon and Greenhouse Gas Group portal of the University of Bremen: \url{https://www.iup.uni-bremen.de/carbon_ghg/products/tropomi_wfmd/}. Each file contains the relevant variables, metadata annotations, and quality flags necessary for the pre-processing and analysis workflows.

\subsubsection{Application in Causal Analysis}
Owing to its fine temporal and spatial resolution, the WFMD XCH\textsubscript{4} dataset is particularly well-suited for causal inference studies in atmospheric methane dynamics. The integration of ancillary environmental and land-cover datasets enables the disambiguation of observed correlations and supports the derivation of causally grounded conclusions. This capability is essential for advancing policy-relevant insights and targeted emission mitigation strategies.


\subsection{Land Cover Data}
The land cover data employed in this study are derived from the "MCD12Q1.061 MODIS Land Cover Type Yearly Global 500m" dataset. This product originates from the Moderate Resolution Imaging Spectroradiometer (MODIS) sensors aboard NASA’s Terra and Aqua satellites. The MCD12Q1 Version 6.1 dataset provides annual global land cover classifications at a spatial resolution of 500 meters, integrating observations from both platforms to improve temporal consistency and spatial coverage. It is produced and maintained by the NASA Land Processes Distributed Active Archive Center (LP DAAC) at the USGS Earth Resources Observation and Science (EROS) Center and is accessible via both the Earth Engine Catalog \cite{modis_lc_ee} and the LAADS DAAC \cite{laads_daac}. This dataset is instrumental for climate, hydrological, and biogeochemical modeling efforts, as it characterizes the spatial heterogeneity of vegetation types and land surface properties on a global scale.

\subsubsection{Key Specifications}
\begin{itemize}
    \item \textbf{Temporal coverage:} 2001 to 2023 (annual cadence).
    \item \textbf{Spatial resolution:} 500 meters per pixel.
    \item \textbf{Geographic coverage:} Global.
    \item \textbf{Data format:} HDF-EOS Grid with multiple science dataset layers (SDSs).
    \item \textbf{Projection:} Sinusoidal projection with global tiling system.
\end{itemize}

The dataset is accessible via Earth Engine: \url{https://developers.google.com/earth-engine/datasets/catalog/MODIS_061_MCD12Q1} and through LAADS DAAC \url{https://ladsweb.modaps.eosdis.nasa.gov}.

\subsubsection{Classification Schemes}
The product includes land cover classifications derived from five major schemes:
\begin{itemize}
    \item \textbf{LC\_Type1:} IGBP (International Geosphere-Biosphere Programme)
    \item \textbf{LC\_Type2:} University of Maryland (UMD)
    \item \textbf{LC\_Type3:} Leaf Area Index/Biome (LAI/FPAR)
    \item \textbf{LC\_Type4:} Biome-Biogeochemical Cycles (BGC)
    \item \textbf{LC\_Type5:} Plant Functional Types (PFT)
\end{itemize}

Each scheme classifies land cover into thematic categories with specific ecological and biogeophysical implications. For example, the IGBP scheme (LC\_Type1) distinguishes 17 land cover types, including forests, savannas, croplands, wetlands, and urban areas. Table~\ref{tab:igbp_classes} lists the IGBP classes and their corresponding integer values.

\begin{table}[htbp]
    \centering
    \caption{IGBP Land Cover Classes (LC\_Type1)}
    \label{tab:igbp_classes}
    \begin{tabular}{ll}
        \toprule
        \textbf{Value} & \textbf{Description} \\
        \midrule
        0 & Water Bodies \\
        1 & Evergreen Needleleaf Forests \\
        2 & Evergreen Broadleaf Forests \\
        3 & Deciduous Needleleaf Forests \\
        4 & Deciduous Broadleaf Forests \\
        5 & Mixed Forests \\
        6 & Closed Shrublands \\
        7 & Open Shrublands \\
        8 & Woody Savannas \\
        9 & Savannas \\
        10 & Grasslands \\
        11 & Permanent Wetlands \\
        12 & Croplands \\
        13 & Urban and Built-Up Lands \\
        14 & Cropland/Natural Vegetation Mosaics \\
        15 & Permanent Snow and Ice \\
        16 & Barren or Sparsely Vegetated \\
        254 & Unclassified (cloud, missing data, etc.) \\
        255 & Fill value \\
        \bottomrule
    \end{tabular}
\end{table}

\subsubsection{Ancillary Layers}
Additional layers provide metadata and quality assurance:
\begin{itemize}
    \item \textbf{Land Cover Property layers (LC\_Prop1 to LC\_Prop3):} FAO LCCS-based assessments of land cover, land use, and surface hydrology.
    \item \textbf{Assessment Confidence Layers (LC\_Prop1\_Assessment to LC\_Prop3\_Assessment):} Confidence scores ranging from 0 to 100\%.
    \item \textbf{QC:} Quality control flags identifying data validity and processing anomalies.
    \item \textbf{LW:} Land/water mask.
\end{itemize}

In this work, we used the LC\_Type1 classification to associate each satellite observation with a representative land cover class. This classification is integrated with methane concentration data from the TROPOMI/WFMD XCH\textsubscript{4} product to analyze spatiotemporal variability across ecological regions. The selected properties used to reproduce the results from \cite{Karoff2023} include:

\begin{itemize}
    \item Seasonal stratification using MODIS-defined seasons:
    \begin{itemize}
        \item \texttt{subtitles} = [\textit{Spring}, \textit{Summer}, \textit{Fall}, \textit{Winter}, \textit{All Seasons}]
    \end{itemize}
    \item Geographic coverage across all major continents:
    \begin{itemize}
        \item \texttt{continents} = [\textit{Africa}, \textit{Asia}, \textit{Australia}, \textit{North America}, \textit{Oceania}, \textit{South America}, \textit{Antarctica}, \textit{Europa}, \textit{World}]
    \end{itemize}
    \item Monthly encoding of seasonal MODIS files:
    \begin{itemize}
        \item \texttt{seasons} = [58108, 58197, 58290, 58384, 58473, 58562, 58655, 58749, 58839, 58928, 59020, 59114]
    \end{itemize}
    \item Land Cover classes used to analyze methane variation across land types:
    \begin{itemize}
        \item \texttt{LC\_LABELS} = [\textit{Evergreen Needleleaf}, \textit{Evergreen Broadleaf}, \textit{Deciduous Needleleaf}, \textit{Deciduous Broadleaf}, \textit{Mixed Forests}, \textit{Closed Shrubland}, \textit{Open Shrubland}, \textit{Woody Savannas}, \textit{Savannas}, \textit{Grasslands}, \textit{Wetlands}, \textit{Croplands}, \textit{Urban}, \textit{Natural Vegetation Mosaic}, \textit{Snow \& Ice}, \textit{Barren}, \textit{Water Bodies}, \textit{Any Land}]
    \end{itemize}
\end{itemize}

\subsubsection{Formulaic Integration}
Let $X_{i}$ denote the land cover class at spatial location $i$ and $Y_{i}$ the corresponding XCH\textsubscript{4} value. We denote the stratified data as
\begin{equation}
\mathcal{D}_c = \{ (X_{i}, Y_{i}) \mid X_{i} = c,\ \forall\ i \in \mathcal{I}_c \},
\end{equation}
where $\mathcal{I}_c$ indexes all observations with class $c$. This stratification forms the basis for comparative statistical and causal inference in downstream analysis.

\subsubsection{Limitations of the MODIS Dataset}
Some limitations of the MCD12Q1.061 dataset include cloud and snow contamination, mixed pixels in transitional biomes, and classification uncertainties in urban, mosaic, or heterogeneous regions. Despite these caveats, the dataset provides a powerful means to analyze spatial trends in methane emissions as a function of land surface processes.

