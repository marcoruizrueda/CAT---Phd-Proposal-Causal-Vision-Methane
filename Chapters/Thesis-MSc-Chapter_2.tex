% #############################################################################
% This is Chapter 2
% !TEX root = ../main.tex
% #############################################################################
% Change the Name of the Chapter i the following line
\fancychapter{Literature Review}
\cleardoublepage
% The following line allows to ref this chapter
\label{chap:back}

%TODO short introduction

This chapter synthesizes the state of the art in causal inference applied to satellite-based environmental monitoring, with a focus on methane (\ch{CH_4}) dynamics. It consolidates methodological advances across five core domains that underpin this thesis: (i) the geophysical behavior and sources of atmospheric methane, (ii) satellite-based retrievals and limitations in current CH\textsubscript{4} products, (iii) statistical causal inference techniques tailored for spatiotemporal environmental data, (iv) the integration of land cover datasets with atmospheric variables, and (v) the baseline empirical analysis by Karoff \& Vara-Vela (2023) \cite{Karoff2023}, which motivates and constrains the present work.

Although existing studies have established robust frameworks for causal inference in economics and biomedicine, only recently have these techniques been adapted for geophysical applications. Within this emerging field, methane has received relatively limited attention despite its critical role in climate feedback loops and its heterogeneous emission sources. This gap is amplified by the challenges of applying causal models to satellite data such as autocorrelation, irregular sampling, and spatial confounding challenges that this research explicitly addresses.

Taken together, the reviewed studies demonstrate a growing interest in applying causal inference to environmental challenges, yet also reveal that methane monitoring from space remains methodologically underserved. The integration of causal analysis with satellite-based observations, particularly when combined with land cover dynamics and atmospheric constituents, remains largely unexplored or limited to correlational approaches. This chapter, therefore, situates the current research within a broader scientific context, highlighting how previous works inform the development of a tailored causal framework for methane attribution. The subsequent sections delve into each component of this interdisciplinary space, providing the necessary foundation for the methodological innovations introduced in the next chapter.


%%%%%%%%%%%%%%%%


\section{Causal Inference Methods for Environmental Time Series}

Traditional environmental analyses have relied on correlation and regression, which cannot distinguish causation from confounding or indirect effects. Judea Pearl's framework for causal inference \cite{Pearl2009} and subsequent advances have enabled the identification of causal relationships from observational data, provided certain assumptions are met.

While remote sensing offers unprecedented data coverage, interpreting complex Earth system relationships requires more than correlation analysis. In time-series analysis, causal inference methods aim to determine whether one variable influences or drives changes in another, moving beyond mere simultaneity. A classical approach is Granger causality (GC), a statistical test introduced by C.W.J. Granger in 1969. In essence, $X$ "Granger-causes" $Y$ if including past values of $X$ significantly improves the prediction of $Y$ compared to using only past $Y$ \cite{Kovacs2023}. In practice, GC is evaluated via predictive linear regression models: if the forecast error of $Y$ is reduced by incorporating lags of $X$, then $X$ is considered a Granger cause of $Y$ (provided certain assumptions like stationarity hold)
\cite{Kovacs2023}. It is important to note that Granger causality, being based on predictability, does not prove a direct physical cause-and-effect; rather, it identifies predictive information flow. Despite this caveat, GC has been widely applied in geosciences as an initial test for directionality in coupled climate variables. For example, recent work by Kovács et al. (2023) \cite{Kovacs2023} employed Granger causality on global satellite indices to untangle drivers of vegetation dynamics. By applying GC analyses pixel-wise between climate variables (like precipitation, temperature) and vegetation indices (NDVI, FAPAR, etc.), they produced global maps of where environmental drivers have causal influence on vegetation anomalies \cite{Kovacs2023}. Such studies demonstrate GC's ability to reveal patterns consistent with known processes (e.g. rainfall Granger-causing vegetation growth in water-limited regions), opening possibilities for improved forecasting \cite{Kovacs2023}. Another pertinent example comes from the paleoclimate realm: Larsson and Persson (2023) \cite{Larsson2023} performed a multivariate Granger causality analysis on 800,000-year ice-core records of temperature, CO$_2$, and CH$_4$. They found strong evidence of bidirectional GC between all three variables, i.e. CO$_2$ Granger-causes temperature and vice versa, and similarly CH$_4$ is both cause and effect in long-term glacial climate feedbacks. This suggests that on geologic timescales, greenhouse gases and temperature mutually reinforce each other in a complex feedback loop. These examples illustrate the utility of Granger causality in climate science, while also reminding us that GC can reflect indirect or common-driver influences (especially when many variables are omitted). 

To capture nonlinear or non-parametric dependencies, information-theoretic measures like Transfer Entropy (TE) have been introduced. Transfer Entropy, proposed by Schreiber (2000) \cite{Schreiber2000}, quantifies the reduction in uncertainty of future $Y$ given knowledge of past $X$ (and past $Y$), effectively measuring the information flow from $X$ to $Y$. Unlike linear GC, transfer entropy can detect nonlinear coupling and does not assume a specific model form. In practice, TE is often estimated through histogram or k-nearest-neighbor methods and requires sufficient data for robust probability distribution estimation. TE has seen growing application in Earth system analysis where relationships may be nonlinear. A noteworthy study by Stips et al. (2016) \cite{Stips2016} used a form of TE (the Liang-Kleeman information flow method \cite{LianKleeman2013}) to investigate the causal links between atmospheric CO$_2$ and global mean temperature \cite{Stips2016}. Their analysis confirmed that, for the modern instrumental period, rising CO$_2$ (and other greenhouse gases) is the causal driver of observed warming, rather than the other way around \cite{Larsson2023}. 

Interestingly, when the same methods were applied to ice-age data, the direction of causality appeared reversed: temperature changes led CO$_2$/CH$_4$ changes, reflecting the slow feedback of carbon cycle processes in paleo-records \cite{Larsson2023}. This example underscores TE's ability to capture directionality in both linear and nonlinear climate interactions. It also highlights the need to interpret causality in context, the causality can be time-scale dependent or mediated by broader system dynamics. More generally, TE and related measures (e.g. conditional mutual information, Rényi transfer entropy) have been used to study topics such as extreme event precursors \cite{Palus2024, Benocci2025}, rainfall dynamics in monsoons \cite{tongal_forecasting_2021}, and even socio-ecological feedbacks \cite{li_integrating_2025}. 

Beyond optimizing computational efficiency for large datasets, recent methodological innovations in causal discovery further strengthen the analytical foundation for rigorously addressing complex environmental challenges like methane attribution. For instance, Cheong et al. \cite{cheong2021causal} demonstrate the power of Structural Causal Models (SCMs) in identifying and correcting for biases. While their specific application was in affect recognition, the core principles of SCMs, involving explicit modeling of confounding structures and the use of structural equations, are highly relevant for disentangling biases and complex dependencies within satellite-derived methane data. 

Similarly, Cheng and Redfern \cite{cheng2021} propose the use of normalized information flow (nIFc) as a robust approach for analyzing high-dimensional Earth system datasets. Their methodology is specifically designed to address challenges analogous to those encountered in methane time series, including complex interdependencies, high noise levels, and sparse observations. These advancements provide more sophisticated tools for unraveling the intricate web of cause-and-effect in environmental systems, moving beyond basic correlative insights.

Recent methodological innovations in causal discovery further strengthen the analytical foundation of this thesis. Cheong et al. \cite{cheong2021causal} demonstrate how structural causal models (SCMs) can be used to identify and correct for bias in affect recognition. While focused on facial expression data, the principles of SCMs, explicit modeling of confounding structures and structural equations are relevant for disentangling biases in satellite, derived methane data. 

Similarly, Cheng and Redfern \cite{cheng2021} propose the use of normalized information flow (nIFc) as a robust approach for quantifying causal contributions in high-dimensional Earth system datasets. Their methodology is specifically designed to provide strong estimates of causal contributions even in challenging situations where standard regressions perform poorly. These situations include cases with significant causal time-lags, substantial noise contributions, alternating feedback (correlation) signs, and complex teleconnections where multiple causes contribute to an overall effect. As exemplified by their analysis of methane-climate feedbacks using both observational data and Earth system models, nIFc offers a powerful means for quantifying nonlinear processes and assessing model performance in contexts directly relevant to methane time series analysis.

Jerzak et al. \cite{jerzak2023} also integrate causal inference on Earth observation data. Their work emphasizes the role of satellite imagery in resolving confounding in observational studies and proposes novel scene-based definitions of treatment and outcome at varying spatial resolutions. These perspectives are especially valuable in methane monitoring, where different processes may operate on spatial scales and resolution mismatches between drivers (e.g., land cover) and outcomes (e.g., XCH\textsubscript{4}) must be causally reconciled.

The CA-SpaceNet framework by Wang et al. \cite{wang_ca-spacenet_2022} offers another example of causal reasoning under complex image-based confounding. Although aimed at space object pose estimation, its use of counterfactual reasoning to reduce background bias parallels the need for bias-resilient causal interpretation in methane remote sensing. 

In the last few years, advanced multivariate causal discovery algorithms have emerged, aiming to unravel complex interdependencies in high-dimensional environmental systems. One prominent example is the PCMCI framework (PCMCI = PC-algorithm for momentary conditional independence), introduced by Runge et al. \cite{Runge2020EGU}. PCMCI combines the classic PC algorithm (which iteratively removes spurious links via conditional independence tests) with a tailored approach for time-lagged data. It performs momentary conditional independence (MCI) tests to identify causal links in specific lags, while accounting for autocorrelation in each time series \cite{Runge2019_2}. This allows PCMCI to handle datasets with many variables and temporal dependencies, typical of Earth system observations. 

In a landmark study, Runge et al. (2019) \cite{Runge2019} demonstrated PCMCI on a dataset with dozens of climate indices and found it could successfully recover known cause-effect relations (like ENSO's influence on Indian monsoon) that simpler correlation or bivariate GC missed \cite{Runge2019, Docquier2024}. 

Subsequent applications in climate science have shown PCMCI's ability to filter out spurious correlations induced by common drivers or autocorrelation, giving a clearer picture of direct causal pathways \cite{Docquier2024}. For instance, Böhnisch et al. (2023) \cite{bohnisch_european_2023} used PCMCI to elucidate the causal network of atmospheric circulation patterns leading to European heat waves, isolating key drivers among many correlated candidates. A recent comparison by Docquier et al. (2024) \cite{Docquier2024} evaluated PCMCI against an information-flow method for North Atlantic climate indices \cite{Docquier2024}. Both methods outperformed simple correlation in identifying true links, with PCMCI excelling when the system involved many variables, thanks to its ability to condition on multiple confounders \cite{Docquier2024}. However, differences emerged in specific link strengths (e.g. one method flagged the Arctic Oscillation as most influential, whereas PCMCI highlighted ENSO) \cite{Docquier2024}. The authors noted that such differences call for further investigation, ideally incorporating nonlinear causal discovery methods in addition to PCMCI's (primarily linear) independence tests \cite{Docquier2024}. This points to an active frontier: hybrid approaches that can capture nonlinearity, handle many variables, and provide confidence measures for causal links in climate data. 

Another important methodological parallel comes from the work of Zhang et al. \cite{zhang_learning_2021}, who developed a bisimulation metric for reinforcement learning that encourages the learning of representations invariant to task-irrelevant details. Although their domain is Markov Decision Processes (MDPs), the core concept—distinguishing signal from noise in high-dimensional environments—translates directly to satellite-based methane analysis. In remote sensing, irrelevant geographic or atmospheric features can obscure causal signals. A causal framework, particularly when used in conjunction with dimensionality-reduction strategies that preserve causal invariants, must be sensitive to these distinctions. The bisimulation approach exemplifies how representation learning techniques can assist in filtering out confounding visual or contextual information in satellite data, potentially enhancing causal interpretability.

In summary, a suite of causal inference tools, from Granger causality and transfer entropy to PCMCI and beyond is now available to probe environmental time series. Each has strengths and weaknesses: GC and pTE offer simplicity and speed but assume linearity (or Gaussianity), whereas TE and information flow capture nonlinearity at the cost of more data and computation, and PCMCI and related algorithms tackle high-dimensional multivariate causality but may miss nonlinear effects. In practice, applying multiple methods in concert can provide complementary insights \cite{Docquier2024, silini_assessing_2023}. The growing literature (especially post-2020) illustrates that these methods, when carefully applied, can successfully disentangle drivers of climate anomalies, ecosystem responses, and other Earth system behaviors that elude standard correlation analysis \cite{Runge2019}. This PhD thesis builds on these developments by applying causal inference to satellite-derived methane time series, a novel combination that leverages the abundant data from remote sensing and the rigor of modern causality frameworks.

Recent years have seen the adaptation of causal discovery methods to environmental time series:
\begin{itemize}
    \item \textbf{Granger causality} \cite{Granger} and its nonlinear extensions have been widely used to infer directional dependencies in climate and ecological data.
    \item \textbf{PCMCI} (Peter and Clark Momentary Conditional Independence) has emerged as a robust method for high-dimensional, autocorrelated time series typical of Earth system datasets \cite{Runge2019}.
    \item \textbf{Transfer entropy} provides a model-free, information-theoretic approach to causality, capturing nonlinear and lagged dependencies \cite{Schreiber2000}.
    \item \textbf{Causality in remote sensing:} \cite{jerzak2023} discuss the challenges and opportunities of integrating Earth observation data into causal inference, highlighting issues such as spatial confounding, measurement error, and the need for domain knowledge.
    \item \textbf{Recent applications:} \cite{Yuan2022} applied causality-constrained machine learning to wetland methane emissions, showing improved interpretability and predictive performance. \cite{Kretschmer2016} and \cite{EbertUphoff2021} demonstrated the utility of causal networks in climate research.
\end{itemize}

Lastly, the increasing integration of machine learning techniques is proving vital for advanced methane monitoring. For instance, Wang et al. \cite{wang2020} developed a machine learning algorithm, leveraging logistic regression, to predict high-emitting sites in the oil and gas sector that could be prioritized for repair. This approach aimed to cost-effectively identify "super-emitters" whose stochastic and intermittent leaks are responsible for a majority of emissions, thereby significantly reducing survey costs and increasing emissions reductions compared to conventional methods.

In the same line, Yuan et al. \cite{Yuan2022} developed a causality-guided machine learning model for wetland \ch{CH4} emissions. This work, using flux measurements from \gls{eddy_covariance} towers, successfully identified consistent causal regulations, such as soil temperature dominating \ch{CH4} emissions across wetland types. The use of ground-level, in-situ devices in such studies is particularly valuable as it allows for a controlled collection of all measured variables, simplifying the co-location and harmonization process, which greatly facilitates robust causal analysis. 

These examples underscore the applicability of causal inference tools beyond climate time series, reinforcing the methodological diversity required to address the methane attribution problem.

\section{Integrating Land Cover, Atmospheric Data, and Causality}
Land cover is a key driver of methane emissions, especially for wetlands, agriculture, and \gls{urban_areas} \cite{Saunois2020, Karoff2023}. MODIS and ESA WorldCover provide global, high-resolution land cover maps that can be integrated with atmospheric data for causal analysis \cite{Friedl2010, Zhang2021}.

A key focus of this research is the integration of land cover information with atmospheric methane data under a causal analysis framework. Land cover and land use have profound effects on methane emissions: for instance, wetlands are the largest natural source of methane, rice paddies and cattle pastures contribute significant agricultural emissions, while urbanized or arid lands emit relatively little CH$_4$. At the same time, atmospheric methane concentrations can feedback on ecosystems (e.g. via climate change). Understanding these interactions requires linking spatio-temporal data on land surface with atmospheric measurements. 

Prior studies have approached this integration mostly through statistical or process-modeling methods rather than explicit causal discovery. One line of research has used remote sensing to connect climate variability, land cover changes, and methane fluxes. Pandey et al. (2017) \cite{Pandey2017} provided an example by examining how an extreme climate event (the 2011 La Ni\~na) impacted wetland extent and thereby methane emissions. They found that anomalously high rainfall during La Ni\~na expanded tropical wetland inundation area, which in turn led to a 5-9 Tg increase in CH$_4$ release, detectable via higher column methane in satellite data. By correlating precipitation anomalies, satellite-derived inundation, and methane observations, their study inferred a possible causal chain: La Ni\~na → increased wetlands → elevated CH$_4$ emissions. Although Pandey et al. did not employ formal causality tests, the evidence strongly suggested that wetlands were the cause of the methane spike (rather than a mere coincidence). This illustrates how multi-source data (climate indices, land cover, gas measurements) can be pieced together to tell a causal story about the Earth system.

At a more local scale, researchers have combined ground-based flux measurements with satellite time series to attribute carbon budget components. In an early study, Yan et al. (2008) \cite{Yan2008} tackled the problem of "closing" the carbon balance in an estuarine wetland by integrating eddy-covariance tower data with MODIS satellite observations. They noted that a simple productivity model severely underestimated the wetland's carbon uptake, hypothesizing that unaccounted methane release and tidal carbon export were causing the discrepancy. By incorporating remote-sensing indicators, a land surface water index (LSWI) for inundation, evapotranspiration (ET) for wetland metabolism, and tidal height, into an autoregressive model of gross primary production (GPP), they significantly improved model performance. The modified model's estimates matched observed net carbon exchange much better ($R^2$ increased from 0.55 to 0.88). Effectively, this approach embedded land surface variables into a time-series prediction, which is closely aligned with causal attribution: it demonstrated that including the effects of water inundation and methane emissions (drivers) was necessary to explain the carbon dynamics. Although Yan et al. did not use the term "causality" their integrated model functioned akin to a causal model, identifying tidal flooding and methane release as drivers of carbon losses that needed to be accounted for. This underscores the value of blending atmospheric data with land cover/phenology data to capture cause-effect relations in biogeochemical cycles.

Most recent studies have also explored data-driven correlations between land use and air quality. Jodhani et al. (2024) \cite{Jodhani2024}, for example, utilized Google Earth Engine to fuse Sentinel-5P pollutant data with land-cover classifications in India. They examined how different LULC categories (urban, agricultural, forest, water, etc.) correspond to variations in gases like CO, NO$_2$, SO$_2$, HCHO, and CH$_4$ over multiple years. Their results showed clear associations for short-lived pollutants (e.g. NO$_2$ hotspots over cities and industrial regions, elevated SO$_2$ near coal power plants). Methane, in contrast, showed a relatively uniform spatial distribution over the region, with no sharp local hotspots. This was attributed to methane's long atmospheric lifetime and well-mixed background, local land use had less immediate impact on column CH$_4$ compared to more regional or global influences. Nonetheless, Jodhani et al. found subtle differences aligned with land cover: for instance, areas of intensive agriculture and wetlands in Gujarat did exhibit slightly higher CH$_4$ concentrations than arid or urban areas (consistent with expected emissions from rice paddies and reservoirs). Studies like this provide a baseline of where methane concentration differences coincide with land surface characteristics. The next step, which our research undertakes, is to rigorously test which differences are causally attributable to land cover as opposed to confounding factors like meteorology or transport.

Integrating land cover data into a causal inference framework for methane involves several strategies. An approach is to stratify the analysis by land cover type. effectively comparing time series of CH$_4$ over different surface categories (wetlands, agriculture, forests, etc.) and testing for lead-lag relationships or causal links between these categories and methane variability. Karoff and Vara-Vela (2023) \cite{Karoff2023} made an initial foray into this by computing separate methane seasonal cycles for each type of land and region. Their study, detailed in Section \ref{sec:baseline}, serves as a baseline for our investigation.

A particularly relevant contribution comes from Jerzak et al. \cite{jerzak2023}, who address the challenge of using satellite imagery as a proxy for confounding variables in causal inference. They discuss the definition of treatment, outcome, and confounder at image scale and examine how resolution mismatches can distort causal estimates. This has strong implications for our integration of land cover (from MODIS/ESA) with methane data (from TROPOMI), where aggregation choices and observational granularity can influence inferred causal links. Their simulation results also guide practical design decisions, such as how kernel bandwidth and pixel size affect bias and variance in estimation.

From a methodological standpoint, integrating land cover in causality analysis often means dealing with spatially distributed time series (maps over time). Techniques like PCMCI have been extended to gridded data, essentially building causal networks in each grid cell and allowing teleconnections between cells \cite{Runge2019, Runge2020EGU}. In our context, rather than analyze every pixel globally (which is computationally prohibitive), we focus on aggregated time series over distinct land cover classes or regions of interest. This retains physical interpretability (e.g. "wetland regions' CH$_4$ vs. upland regions' CH$_4$") while simplifying the causal discovery problem. The literature suggests this approach is promising: for example, Kovács et al. (2023) \cite{Kovacs2023} derived causal links between global biomes and climate drivers by aggregating pixels into biome categories, finding that precipitation Granger-caused vegetation changes most strongly in arid biomes \cite{Kovacs2023}. Analogously, we might find that certain land-cover types (like wetlands) exert a causal influence on methane anomalies (e.g. via episodic release events) whereas others (like deserts) do not, or that the causal effect is mediated by climate variables. 


To our knowledge, no prior study has systematically applied causal discovery algorithms to satellite-based methane time series categorized by land cover, within an automated framework. Thus, our work pioneers this integration, aiming to identify cause-effect relations such as "wetland extent → methane concentration" or "irrigated cropland → methane" in a data-driven way. This fills a gap between remote-sensing studies that correlate CH$_4$ with land surface properties and process-based modeling studies that simulate CH$_4$ emissions given land surface conditions. Using causal methods, we seek to move from correlation to attribution, determining which land cover factors are genuine drivers of methane variability and which observed associations might be spurious or indirectly caused by other factors.


\section{Baseline Study by Karoff \& Vara-Vela (2023)} 
\label{sec:baseline}

The analysis by Karoff and Vara-Vela (2023) \cite{Karoff2023} serves as an important starting point for our research. Their study is a data-driven exploration of how atmospheric methane concentrations vary as a function of geography, land cover type, and season. In many ways, it establishes the empirical patterns and puzzles that a causal study (like ours) needs to explain. Karoff and Vara-Vela compiled three years of TROPOMI Sentinel-5P XCH$_4$ data (2017-2020) and stratified the observations by land cover categories using MODIS and ESA WorldCover datasets. They deliberately processed the methane data with two independent retrieval algorithms (as noted earlier, one of which was likely the WFMD/WFM-DOAS and the other the RemoTeC-based product) to ensure that results were not an artifact of a single retrieval method. The methane measurements were then aggregated over each land cover class (such as croplands, wetlands, forests, shrublands, \gls{savannas}, urban areas, etc.) and analyzed for differences in mean concentration and seasonal cycle.

In general, they observed that the phase of the annual CH$_4$ cycle differs by continent: over croplands, shrublands, and savannas, African sites showed a delayed seasonal peak compared to Asian sites. This hints that regional climatic differences (e.g. timing of wet/dry seasons) modulate methane seasonality depending on land context. A causal analysis could build on this by including climate variables (temperature, precipitation) along with land type as potential causes of the methane seasonal phase shift. Another strategy is using causal graph or network models where nodes include not just CH$_4$ concentration, but also land-surface variables like normalized difference wetness index (for wetlands), burned area (for biomass burning), or anthropogenic activity proxies. By treating land indicators as part of a multivariate causal model, one can ask, for instance: Do changes in wetland extent Granger-cause changes in methane, after controlling for precipitation and temperature? Or Does irrigation expansion (cropland increase) in a region lead to rising CH$_4$ levels, or are both driven by a third factor? There is nascent work in this direction, such as studies linking deforestation to local climate via GC tests \cite{Kovacs2023}, but applying it specifically to methane remains largely unexplored.

The authors' findings reported that, on a global scale, the highest column-average methane tends to occur over agricultural areas, especially croplands, whereas the lowest methane values occur over regions like shrublands. This aligns with expectations: intensive agriculture (including rice paddies and ruminant livestock) is a major anthropogenic methane source, whereas shrublands (often arid or semi-arid) have sparse vegetation and low wetland extent, hence fewer methane emitters. Such a gradient from high CH$_4$ over croplands to low CH$_4$ over shrublands was consistently observed on multiple continents. In contrast, forests and savannas showed intermediate methane levels, and interestingly, urban areas did not stand out as high-methane regions (likely because methane sources in cities are relatively diffuse or small compared to the wide-area averaging of satellite pixels). The most surprising result was that methane concentrations over wetlands were not as high as expected. 

In fact, Karoff \& Vara-Vela found that wetlands exhibited lower than average CH$_4$ concentrations in the satellite data. This is counter-intuitive since wetlands are well-known natural methane sources (due to anaerobic decomposition in waterlogged soils). One would anticipate wetlands to show elevated CH$_4$ column abundance. The authors noted this discrepancy and cautioned that methane measurements from TROPOMI over wetland areas should be "handled with caution" until the cause is understood. They hypothesized that retrieval issues (perhaps related to water, humidity, or aerosol in wetland regions) might be contributing to anomalously low satellite readings, rather than wetlands truly having less methane. This unresolved point directly motivates a deeper investigation, potentially with causal analysis to discern if the low wetland CH$_4$ is a retrieval artifact (instrumental cause) or if there is an actual physical mechanism (e.g. wetlands coinciding with strong OH radical presence that destroys methane, though such an effect would be unusual). Our work will build on this finding by probing it with causal discovery tools, to see if we can detect any systematic "bias" link between wetland fraction and CH$_4$ (which would support the artifact theory) or any external driver that depresses CH$_4$ over wetlands.

In addition to spatial patterns, Karoff \& Vara-Vela examined seasonal variations of methane over different land types. They observed that methane generally has a seasonal cycle (driven by factors like hydroxyl radical seasonality and emissions timing), and this cycle's phase can differ by region. For example, over croplands, shrublands, and savannas in Africa, the annual peak in CH$_4$ occurred later in the year compared to similar land covers in Asia. One interpretation is that African methane sources (and/or transport patterns) are phased differently, possibly related to later wet seasons or biomass burning events in the Southern Hemisphere. While their research paper did not conclusively explain these phase shifts, it highlighted that geographic context matters: the interplay of land cover with regional climate leads to unique methane timing signatures. This again suggests the presence of underlying causal factors (monsoonal rains, temperature cycles, etc.) that differ between continents. Karoff \& Vara-Vela's study was intentionally kept data-driven and non-causal, they mapped and described patterns, without formally attributing causes. They emphasized known expectations (like cropland vs shrubland differences) and flagged anomalies (like the wetlands issue), but stopped short of testing hypotheses about why those patterns occur. In their conclusion, they specifically pointed out the wetland methane discrepancy as an open question. They also mentioned that until such discrepancies are resolved, one should be careful in using the data for emission quantification. This cautious approach underscores the need for further analysis, precisely what our thesis aims to do by applying causal inference techniques.

In summary, the Karoff \& Vara-Vela (2023) study provides a valuable benchmark of methane distribution across land covers, against which we can compare our causal analysis results. It represents the state-of-the-art in pure statistical description of satellite methane data. Our work extends this by inferring directionality and drivers: for instance, where they noted high CH$_4$ over croplands, we will investigate whether cropland extent Granger-causes methane increases (indicating emissions effect) or if the correlation could be due to a third factor. Where they found the puzzling wetland low bias, we will use causal methods to test potential explanations (such as a hidden variable like aerosol optical depth affecting the retrievals). In doing so, we treat Karoff \& Vara-Vela's results as the foundation, a non-causal baseline, upon which a causal understanding will be built. This progression from description to causation exemplifies the evolution of scientific inquiry in this domain. By positioning their study as the baseline, we ensure that our advanced analyses remain grounded in the empirical reality of the observations, and we directly address the gaps they identified.


\section{Knowledge Gaps, Challenges, and Research Opportunities}

Reviewing the literature reveals several critical knowledge gaps at the intersection of satellite methane observation, land cover characterization, and causal analysis. First, while numerous studies have leveraged satellite data to correlate methane with land surface features or external drivers (ENSO events, seasonal cycles, etc.), very few have applied formal causal inference techniques to these datasets. The years 2020-2024 saw a surge in Earth science causality research \cite{Runge2019}, yet applications have mostly centered on climate indices or model outputs, not on remotely-sensed greenhouse gas measurements. In particular, there is a lack of studies that take the rich spatiotemporal record of methane from TROPOMI and ask causally (e.g., what drives the observed variations and trends across different landscapes). 

The baseline analysis by Karoff \& Vara-Vela (2023) \cite{Karoff2023} is emblematic as this mapped differences but did not attribute cause nor formal correlationships. As a result, some observations are unexplained (e.g. wetlands show anomalously low CH$_4$). Is it a data artifact?, or could it be that wetlands, perhaps being largely remote and clean-air sites, coincide with conditions that favor methane oxidation (for instance, higher UV leading to more OH radicals)?. To date, no studies have conclusively answered this question. A targeted causal analysis could test hypotheses by including potential explanatory variables (such as retrieval parameters, cloud cover, \ch{OH} concentration proxies) to see if they explain the wetland anomaly.

Another gap lies in understanding spatiotemporal variability of methane in relation to land cover changes. We know qualitatively that if, say, wetlands expand, methane emissions likely rise, and if wetlands dry up or are drained, emissions fall. However, this has rarely been demonstrated with observational data in a causal sense. The Pandey et al. (2017) \cite{Pandey2017} study gave one example with ENSO-driven wetland expansion, but such events are rare. More common are gradual land use changes, for instance, agricultural expansion, urban sprawl, or natural wetland loss; happening continuously around the globe. Do these changes leave a discernible fingerprint in atmospheric methane growth rates regionally? If so, can we detect it with Granger causality or PCMCI by linking land cover time series to methane time series? This remains an open question. 


The current literature on land-atmosphere causal links has focused more on CO$_2$ (for example, deforestation effects on local climate) or on weather extremes and land cover \cite{Kovacs2023}. Methane, being a less abundant but more potent greenhouse gas, has not received the same causal scrutiny. With new data available (e.g. annual land cover maps from MODIS or Copernicus and daily methane maps from TROPOMI), we are now in a position to investigate these links. A potential outcome could be identifying "hotspots" where land use change is actually causing detectable methane concentration changes (e.g., rapid irrigation development causing more rice paddies and thus methane, or conversely, wetland restoration reducing methane if the restored wetlands are managed in a certain way).

%%%%%%%%%% finished here
A further gap is methodological: bridging the scales between local process studies and global satellite analysis. Many causal inference techniques (including PCMCI, TE) were originally developed for well-defined time series (like climate indices or station data). Applying them to gridded satellite data involves challenges like high dimensionality, autocorrelation across space, and the confounding effect of atmospheric transport that mixes signals from different regions. The Nature Communications perspective by Runge et al. (2019) \cite{Runge2019} emphasizes the need to tailor causal discovery methods to Earth science data, warning that naive application can lead to misinterpretation \cite{Runge2019}. The community has responded by creating platforms like CauseMe.net \cite{Runge2019, Runge2019_2, MunozMari2020} for benchmarking methods on realistic datasets. However, in the specific context of methane and land cover, there is virtually no prior art, meaning we must navigate issues like choosing appropriate spatial aggregation (to avoid unwieldy variable counts), detrending for seasonal cycles (to avoid spurious causality from common trends), and ensuring the data meet assumptions of the causal tests (e.g. approximate stationarity after removing known trends). Our literature review did not find any study that has attempted PCMCI or Granger tests on, say, time series of "methane over forests" vs "methane over croplands" or similar. This represents a methodological gap our work aims to fill, by developing a framework for such analysis.

Another knowledge gap highlighted by the literature is the interpretation of causality in complex systems. Even when causal links are identified, making physical sense of them can be challenging. For example, Docquier et al. (2024) \cite{Docquier2024} found differing causal drivers for climate indices depending on the method \cite{Docquier2024}. In our context, suppose we find that "cropland fraction Granger-causes methane" in certain regions. We must interpret whether this truly means agricultural emissions are driving methane increases (which is plausible), or whether cropland fraction is acting as a proxy for some other factor (e.g. human population or industrial activity that co-occurs with croplands). Similarly, if we find no causal link for wetlands, does that mean wetlands don't influence atmospheric methane? Or could it be that the satellite data's noise masks the link, or that the causal effect is non-linear (e.g. only when wetlands exceed a certain inundation threshold do emissions spike, which a linear test might miss)? These nuances imply a gap between statistical causality detection and domain understanding. The literature suggests combining domain knowledge with data-driven results to avoid misattribution \cite{silini_assessing_2023, Runge2019}. In our thesis, we plan to address this by cross-checking causal findings with independent evidence (e.g. emission inventories, known event timelines like droughts or floods, etc.). For instance, if a causal analysis flags a particular land cover as a driver of methane in a region, we will verify whether known emission inventory data or field studies support that. This step is necessary to translate the quantitative causal discovery into credible scientific conclusions.

Finally, the confluence of these topics (satellite methane, land cover, and causality) points to a gap in practical applications for climate change mitigation. Satellites like Sentinel-5p were designed not just for science but also to support emissions mitigation by identifying sources. To date, most efforts in that vein have focused on pinpointing "super-emitters" (e.g. major gas leaks) via remote sensing. The broader question of causal attribution (e.g. which sectors or land uses are most responsible for observed methane increases) is often addressed using bottom-up inventories or inverse modeling, rather than direct data-driven causality analysis. Our literature survey indicates that applying causal inference directly on observations could provide a complementary, data-driven perspective to these top-down model studies. For example, if agricultural land expansion in South Asia is causally linked with methane increases, this reinforces inventory-based attributions and could sharpen policy focus. Conversely, if certain wetland regions show rising methane that is not causally explained by local temperature or precipitation changes, it might hint at unaccounted anthropogenic inputs or biases. Such insights would be highly relevant to closing the global methane budget, which still has uncertainties (often termed the "methane enigma" for the post-2007 renewed rise). By identifying which variables and regions play a causal role, we contribute to narrowing those knowledge gaps.

% TODO: Merge the section with the following:
Despite notable progress in applying causal analysis to Earth observation data, significant research gaps and methodological challenges remain challenges that our work directly aims to address:

\paragraph{Focus on Correlation over Causation:} 
Most prior studies in methane remote sensing and climate data analysis focus on identifying correlations or temporal trends, without conclusively establishing causal relationships. There is a lack of systematic causal inference applied specifically to methane observations, a gap this study addresses by centering the analysis on explicit cause-effect questions rather than descriptive associations.

\paragraph{Limited Integration of Data Sources.} 
Few existing works integrate multiple satellite products and land surface data within a unified causal framework. Typically, studies examine one dataset or one potential driver at a time. In contrast, our approach simultaneously considers multiple atmospheric variables (e.g., \ch{CO}, \ch{NO2}), land cover types, and meteorological context. This enables us to uncover multifactor causal networks, such as how land cover might modulate the influence of temperature on methane levels. Such data fusion is relatively novel and allows for the discovery of interactions that single-source analyses may overlook.

\paragraph{Handling of Confounding and Autocorrelation:} 
Many existing studies insufficiently address key methodological issues, such as confounding factors, autocorrelation in time series, and the multiple comparisons problem. Climate studies often rely on simple lag correlations, which can be misleading if autocorrelation is not properly accounted for, or if multiple lags are tested without correction. We explicitly incorporate techniques to address these issues: multivariate testing for confounding, autocorrelation-aware methods such as PCMCI, and statistical correction procedures like false discovery rate (FDR) control. This rigorous methodology aims to reduce false positives and increase confidence in the robustness of inferred causal links.

\paragraph{Reproducibility and Open Tools:} 
There is a marked lack of reproducible and openly available workflows for causal inference in the context of environmental remote sensing. Much of the work in this area relies on bespoke scripts that are not easily generalizable or reusable. This represents a clear opportunity for methodological innovation. A core contribution of our project is the development and publication of an open-source causal analysis framework, complete with code and documentation, which can be adopted or extended by other researchers. By promoting transparency and reuse, this work moves the community toward more standardized and collaborative practices in causal inference for Earth system science.

\medskip

By identifying and targeting these research gaps, our project clarifies key opportunities for advancing the field. It will demonstrate how moving from correlation to causation can yield deeper scientific insights, for example, by distinguishing true drivers of methane variability from spurious associations. It will also illustrate the power of data fusion: integrating atmospheric and land surface data under a causal lens to uncover previously unrecognized mechanisms (e.g., specific land cover classes exerting influence on methane seasonal amplitudes). Finally, by openly sharing our framework, we aim to catalyze further research, enabling its application to other greenhouse gases or environmental processes and thereby amplifying the scientific impact of our work beyond the specific case of methane.

In conclusion, this literature review underscores the novelty and importance of our research focus. The gaps define the frontier that this work aims to push forward. Through careful synthesis of recent advances and foundational studies, we have formulated a research approach that uses state-of-the-art causal inference to tackle open questions left by prior data-driven analyses. The ensuing chapters will detail our methodology to address these gaps and will compare our findings against the expectations and puzzles drawn from the literature. In doing so, we hope to contribute both to a new scientific understanding of methane dynamics and a demonstration of how modern causal tools can be applied in Earth observation research \cite{Runge2019}.

%%%%%



